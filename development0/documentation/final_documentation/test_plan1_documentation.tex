\begin{center}\rule{0.5\linewidth}{\linethickness}\end{center}

5.1.4 Receiver Operating Characteristic curve

Receiver Operating Characteristic (ROC) is a curve in a unit square
which is used to assess the accuracy of the diagnostic (e.g.~rating)
system. The area under curve (AUC) is the area under the ROC curve.
Higher AUC means higher discriminatory power of model.

\begin{longtable}[]{@{}ll@{}}
\toprule
Area under curve & Gini coefficient\tabularnewline
\midrule
\endhead
0.7743 & 0.5487\tabularnewline
\bottomrule
\end{longtable}

As shown in the above table, the AUC value is higher than 0.7 and
correspondingly the Gini value of the model is higher than 0.4,
indicating that discriminatory power of the model is satisfactory.

\begin{center}\rule{0.5\linewidth}{\linethickness}\end{center}

5.1.5 KS statistic

The KS statistic is used to measure the discriminatory power of the PD
model. It is defined as the maximum difference between the cumulative
percentage of good samples (i.e., non-defaulters) and the cumulative
percentage of bad samples (i.e., defaulters). A higher KS value implies
a good fit of the model.

The test returns a KS value.

\begin{longtable}[]{@{}ll@{}}
\toprule
Ks statistic train & Ks statistic test\tabularnewline
\midrule
\endhead
0.4219 & 0.3811\tabularnewline
\bottomrule
\end{longtable}

As shown in the above table, the KS value is higher than 0.4, indicating
that discriminatory power of the model is good.

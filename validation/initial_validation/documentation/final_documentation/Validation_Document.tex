``Bank Name''

MODEL VALIDATION MODEL RISK MANAGEMENT

Annual Validation Report

CRE loan balances - Probability of Default Model (Model ID 894 Version
3.2)

Prepared by:

First Name, Last Name

Title, Model Validation (MV)

Reviewed by:

First Name, Last Name

Title, Model Validation (MV)

Identification

ID

Model Version Number

Model

File Name

894

3.2

CRE loan balances - Probability of Default model

Annual Validation

Table of Contents

1 EXECUTIVE SUMMARY

2 INTRODUCTION

2.1 Model source

2.2 Model composition

2.3 Model version

2.4 Model usage and product definition

2.5 Validation coverage

3 MODEL DESCRIPTION

3.1 Model overview

3.2 Theoretical background

3.3 Significant assumptions and limitations

3.4 Model input

3.5 Model output

4 MV METHODOLOGY ASSESSMENT

4.1 Literature review

4.2 Conceptual soundness

4.2.1 Model Framework and Calculation Method

4.2.2 Alternate Methodologies

4.2.3 Model Segmentation

4.2.4 Model Driver/Variable Selection

4.2.5 COVID-19 impact assessment

4.2.6 Review of any outstanding regulatory or IA findings, if applicable

4.3 Assesment of input data

4.4 Bias and fairness

4.5 Assesment of significant assumptions and limitations

4.6 Assesment of source of uncertainty and margin of conservatism

4.7 Assesment of the documentation

4.8 Assesment of performance monitoring approach

4.9 Vendor model contingency plan (if applicable, otherwise delete)

5 MODEL PERFORMANCE ASSESSMENT

5.1 PD model replication

5.1.1 Data quality

5.1.2 Univariate analysis

5.1.3 Model coefficient comparison

5.2 PD model stress test

5.2.1 Kolmogorov--Smirnov test

5.2.2 Variance Inflation Factor

5.2.3 Receiver Operating Characteristic curve

5.2.4 Normal test

5.3 Conclusions and recommendations

5.4 Model limitations

5.5 Model risk assessment

6 REFERENCES

7 APPENDIX

7.1 Final MDR

7.2 Other MV materials

8 ASSESSMENT OF REGULATORY REQUIREMENTS

9 ISSUE LOG

\begin{enumerate}
\def\labelenumi{\arabic{enumi}.}
\setcounter{enumi}{4}
\tightlist
\item
  Model Performance Assessment
\end{enumerate}

MV performed an annual validation of the CRE loan balances PD model. The
model is developed using logistic regression technique with 7
independent variables. Data was sourced from TREPP and was split into
training and testing data by random sampling approach. The development
of the model was in Python, however the validation is performed in R.

In the current validation exercise, MV has performed the following
activities for the selective varibles: 1.) Replication of the PD model -
univariate analysis of risk drivers, comparison of independently
replicated model coefficients with values as provided by development
team 2.) Validation testing of the PD model - KS test, ROC curve, VIF
and Normal test have been performed by the MV

The definations of the selected variables are listed below:

priorfydscr:Preceding Fiscal Year DSCR (NOI). A ratio of net operating
income (NOI) to debt service for the most recent fiscal year end
statement available as reported by the servicer.

mrfytdocc: Most Recent Occupancy Rate. The most recent available
percentage of rentable space occupied. Should be derived from a rent
roll or other document indicating occupancy consistent with most recent
documentation.

debt\_yield\_p1: Debt yield of preceding financial year.

priorfyocc:Preceding Fiscal Year Occupancy Rate. A percentage of
rentable space occupied as of the most recent fiscal year end operating
statement available. Should be derived from a rent roll or other
document indicating occupancy, and in most cases should be within 45
days of the most recent fiscal year end financial statement.

OLTV:LTV at the time of origination.

Division: Division created from the states of US.

interestonly:Interest Only (Y/P/N). If loan is interest only for life,
then Y; if loan has more than one interest only period but is not fully
interest only then P; else N.

The details of the above are provided in the sections below.

\begin{center}\rule{0.5\linewidth}{\linethickness}\end{center}

5.1 PD model replication

5.1.1 Data quality

In data quality analysis, we focussed on assessing whether the data used
in input of the model (e.g., data from systems and databases) are
reliable. Following tests are performed: 1) Missing values 2)
Descriptive statistics

Missing values

\begin{longtable}[]{@{}ll@{}}
\toprule
Feature & Missing\tabularnewline
\midrule
\endhead
mrfytdocc & 42.87\%\tabularnewline
priorfyocc & 25.28\%\tabularnewline
priorfydscr & 21.82\%\tabularnewline
debt\_yield\_p1 & 17.51\%\tabularnewline
OLTV & 3.4\%\tabularnewline
\bottomrule
\end{longtable}

Descriptive statistics

\begin{longtable}[]{@{}lllllllll@{}}
\toprule
\begin{minipage}[b]{0.05\columnwidth}\raggedright
Metrics\strut
\end{minipage} & \begin{minipage}[b]{0.08\columnwidth}\raggedright
Priorfydscr\strut
\end{minipage} & \begin{minipage}[b]{0.09\columnwidth}\raggedright
Mrfytdocc\strut
\end{minipage} & \begin{minipage}[b]{0.09\columnwidth}\raggedright
Debt yield p1\strut
\end{minipage} & \begin{minipage}[b]{0.08\columnwidth}\raggedright
Priorfyocc\strut
\end{minipage} & \begin{minipage}[b]{0.09\columnwidth}\raggedright
Oltv\strut
\end{minipage} & \begin{minipage}[b]{0.09\columnwidth}\raggedright
Division\strut
\end{minipage} & \begin{minipage}[b]{0.08\columnwidth}\raggedright
Interestonly\strut
\end{minipage} & \begin{minipage}[b]{0.11\columnwidth}\raggedright
Bad flag final v3\strut
\end{minipage}\tabularnewline
\midrule
\endhead
\begin{minipage}[t]{0.05\columnwidth}\raggedright
count\strut
\end{minipage} & \begin{minipage}[t]{0.08\columnwidth}\raggedright
72687\strut
\end{minipage} & \begin{minipage}[t]{0.09\columnwidth}\raggedright
53116\strut
\end{minipage} & \begin{minipage}[t]{0.09\columnwidth}\raggedright
76691\strut
\end{minipage} & \begin{minipage}[t]{0.08\columnwidth}\raggedright
69471\strut
\end{minipage} & \begin{minipage}[t]{0.09\columnwidth}\raggedright
89812\strut
\end{minipage} & \begin{minipage}[t]{0.09\columnwidth}\raggedright
92974\strut
\end{minipage} & \begin{minipage}[t]{0.08\columnwidth}\raggedright
92974\strut
\end{minipage} & \begin{minipage}[t]{0.11\columnwidth}\raggedright
92974\strut
\end{minipage}\tabularnewline
\begin{minipage}[t]{0.05\columnwidth}\raggedright
unique\strut
\end{minipage} & \begin{minipage}[t]{0.08\columnwidth}\raggedright
N/A\strut
\end{minipage} & \begin{minipage}[t]{0.09\columnwidth}\raggedright
N/A\strut
\end{minipage} & \begin{minipage}[t]{0.09\columnwidth}\raggedright
N/A\strut
\end{minipage} & \begin{minipage}[t]{0.08\columnwidth}\raggedright
N/A\strut
\end{minipage} & \begin{minipage}[t]{0.09\columnwidth}\raggedright
N/A\strut
\end{minipage} & \begin{minipage}[t]{0.09\columnwidth}\raggedright
10\strut
\end{minipage} & \begin{minipage}[t]{0.08\columnwidth}\raggedright
3\strut
\end{minipage} & \begin{minipage}[t]{0.11\columnwidth}\raggedright
N/A\strut
\end{minipage}\tabularnewline
\begin{minipage}[t]{0.05\columnwidth}\raggedright
top\strut
\end{minipage} & \begin{minipage}[t]{0.08\columnwidth}\raggedright
N/A\strut
\end{minipage} & \begin{minipage}[t]{0.09\columnwidth}\raggedright
N/A\strut
\end{minipage} & \begin{minipage}[t]{0.09\columnwidth}\raggedright
N/A\strut
\end{minipage} & \begin{minipage}[t]{0.08\columnwidth}\raggedright
N/A\strut
\end{minipage} & \begin{minipage}[t]{0.09\columnwidth}\raggedright
N/A\strut
\end{minipage} & \begin{minipage}[t]{0.09\columnwidth}\raggedright
South-Atlantic\strut
\end{minipage} & \begin{minipage}[t]{0.08\columnwidth}\raggedright
N\strut
\end{minipage} & \begin{minipage}[t]{0.11\columnwidth}\raggedright
N/A\strut
\end{minipage}\tabularnewline
\begin{minipage}[t]{0.05\columnwidth}\raggedright
freq\strut
\end{minipage} & \begin{minipage}[t]{0.08\columnwidth}\raggedright
N/A\strut
\end{minipage} & \begin{minipage}[t]{0.09\columnwidth}\raggedright
N/A\strut
\end{minipage} & \begin{minipage}[t]{0.09\columnwidth}\raggedright
N/A\strut
\end{minipage} & \begin{minipage}[t]{0.08\columnwidth}\raggedright
N/A\strut
\end{minipage} & \begin{minipage}[t]{0.09\columnwidth}\raggedright
N/A\strut
\end{minipage} & \begin{minipage}[t]{0.09\columnwidth}\raggedright
19311\strut
\end{minipage} & \begin{minipage}[t]{0.08\columnwidth}\raggedright
85294\strut
\end{minipage} & \begin{minipage}[t]{0.11\columnwidth}\raggedright
N/A\strut
\end{minipage}\tabularnewline
\begin{minipage}[t]{0.05\columnwidth}\raggedright
mean\strut
\end{minipage} & \begin{minipage}[t]{0.08\columnwidth}\raggedright
1.8395755298\strut
\end{minipage} & \begin{minipage}[t]{0.09\columnwidth}\raggedright
93.4809228426\strut
\end{minipage} & \begin{minipage}[t]{0.09\columnwidth}\raggedright
12.5337343705\strut
\end{minipage} & \begin{minipage}[t]{0.08\columnwidth}\raggedright
93.754880108\strut
\end{minipage} & \begin{minipage}[t]{0.09\columnwidth}\raggedright
66.4377253596\strut
\end{minipage} & \begin{minipage}[t]{0.09\columnwidth}\raggedright
N/A\strut
\end{minipage} & \begin{minipage}[t]{0.08\columnwidth}\raggedright
N/A\strut
\end{minipage} & \begin{minipage}[t]{0.11\columnwidth}\raggedright
0.011960333\strut
\end{minipage}\tabularnewline
\begin{minipage}[t]{0.05\columnwidth}\raggedright
std\strut
\end{minipage} & \begin{minipage}[t]{0.08\columnwidth}\raggedright
1.8170684715\strut
\end{minipage} & \begin{minipage}[t]{0.09\columnwidth}\raggedright
9.0892893303\strut
\end{minipage} & \begin{minipage}[t]{0.09\columnwidth}\raggedright
13.8366159547\strut
\end{minipage} & \begin{minipage}[t]{0.08\columnwidth}\raggedright
9.2084476055\strut
\end{minipage} & \begin{minipage}[t]{0.09\columnwidth}\raggedright
17.2802191395\strut
\end{minipage} & \begin{minipage}[t]{0.09\columnwidth}\raggedright
N/A\strut
\end{minipage} & \begin{minipage}[t]{0.08\columnwidth}\raggedright
N/A\strut
\end{minipage} & \begin{minipage}[t]{0.11\columnwidth}\raggedright
0.1087079139\strut
\end{minipage}\tabularnewline
\begin{minipage}[t]{0.05\columnwidth}\raggedright
min\strut
\end{minipage} & \begin{minipage}[t]{0.08\columnwidth}\raggedright
-2.8302\strut
\end{minipage} & \begin{minipage}[t]{0.09\columnwidth}\raggedright
0.78\strut
\end{minipage} & \begin{minipage}[t]{0.09\columnwidth}\raggedright
-26.38654378\strut
\end{minipage} & \begin{minipage}[t]{0.08\columnwidth}\raggedright
1\strut
\end{minipage} & \begin{minipage}[t]{0.09\columnwidth}\raggedright
0.9\strut
\end{minipage} & \begin{minipage}[t]{0.09\columnwidth}\raggedright
N/A\strut
\end{minipage} & \begin{minipage}[t]{0.08\columnwidth}\raggedright
N/A\strut
\end{minipage} & \begin{minipage}[t]{0.11\columnwidth}\raggedright
0\strut
\end{minipage}\tabularnewline
\begin{minipage}[t]{0.05\columnwidth}\raggedright
25\%\strut
\end{minipage} & \begin{minipage}[t]{0.08\columnwidth}\raggedright
1.2978\strut
\end{minipage} & \begin{minipage}[t]{0.09\columnwidth}\raggedright
91.9175\strut
\end{minipage} & \begin{minipage}[t]{0.09\columnwidth}\raggedright
6.8698379192\strut
\end{minipage} & \begin{minipage}[t]{0.08\columnwidth}\raggedright
92\strut
\end{minipage} & \begin{minipage}[t]{0.09\columnwidth}\raggedright
62.8\strut
\end{minipage} & \begin{minipage}[t]{0.09\columnwidth}\raggedright
N/A\strut
\end{minipage} & \begin{minipage}[t]{0.08\columnwidth}\raggedright
N/A\strut
\end{minipage} & \begin{minipage}[t]{0.11\columnwidth}\raggedright
0\strut
\end{minipage}\tabularnewline
\begin{minipage}[t]{0.05\columnwidth}\raggedright
50\%\strut
\end{minipage} & \begin{minipage}[t]{0.08\columnwidth}\raggedright
1.53\strut
\end{minipage} & \begin{minipage}[t]{0.09\columnwidth}\raggedright
95.1\strut
\end{minipage} & \begin{minipage}[t]{0.09\columnwidth}\raggedright
10.461944328\strut
\end{minipage} & \begin{minipage}[t]{0.08\columnwidth}\raggedright
96\strut
\end{minipage} & \begin{minipage}[t]{0.09\columnwidth}\raggedright
72.59\strut
\end{minipage} & \begin{minipage}[t]{0.09\columnwidth}\raggedright
N/A\strut
\end{minipage} & \begin{minipage}[t]{0.08\columnwidth}\raggedright
N/A\strut
\end{minipage} & \begin{minipage}[t]{0.11\columnwidth}\raggedright
0\strut
\end{minipage}\tabularnewline
\begin{minipage}[t]{0.05\columnwidth}\raggedright
75\%\strut
\end{minipage} & \begin{minipage}[t]{0.08\columnwidth}\raggedright
1.86\strut
\end{minipage} & \begin{minipage}[t]{0.09\columnwidth}\raggedright
100\strut
\end{minipage} & \begin{minipage}[t]{0.09\columnwidth}\raggedright
14.0404936145\strut
\end{minipage} & \begin{minipage}[t]{0.08\columnwidth}\raggedright
100\strut
\end{minipage} & \begin{minipage}[t]{0.09\columnwidth}\raggedright
77.6\strut
\end{minipage} & \begin{minipage}[t]{0.09\columnwidth}\raggedright
N/A\strut
\end{minipage} & \begin{minipage}[t]{0.08\columnwidth}\raggedright
N/A\strut
\end{minipage} & \begin{minipage}[t]{0.11\columnwidth}\raggedright
0\strut
\end{minipage}\tabularnewline
\begin{minipage}[t]{0.05\columnwidth}\raggedright
max\strut
\end{minipage} & \begin{minipage}[t]{0.08\columnwidth}\raggedright
63.97\strut
\end{minipage} & \begin{minipage}[t]{0.09\columnwidth}\raggedright
100\strut
\end{minipage} & \begin{minipage}[t]{0.09\columnwidth}\raggedright
408.3590591\strut
\end{minipage} & \begin{minipage}[t]{0.08\columnwidth}\raggedright
100\strut
\end{minipage} & \begin{minipage}[t]{0.09\columnwidth}\raggedright
96.24\strut
\end{minipage} & \begin{minipage}[t]{0.09\columnwidth}\raggedright
N/A\strut
\end{minipage} & \begin{minipage}[t]{0.08\columnwidth}\raggedright
N/A\strut
\end{minipage} & \begin{minipage}[t]{0.11\columnwidth}\raggedright
1\strut
\end{minipage}\tabularnewline
\bottomrule
\end{longtable}

5.1.2 Univariate analysis

In a univariate analysis, we set up a multitude of logistic regression
models where there is only one explanatory variable (the risk driver)
and the response variable (the default). Among those models, the ones
that describes best the response variable can indicate the most
significant explanatory variables, which can then be used to perform a
multivariate analysis.

\begin{longtable}[]{@{}llll@{}}
\toprule
Parameter & Coef & P z & Gini coefficient\tabularnewline
\midrule
\endhead
priorfydscr & -0.881726911 & 0 & 0.452\tabularnewline
mrfytdocc & -1.2194863213 & 0 & 0.321\tabularnewline
debt\_yield\_p1 & -0.8919738737 & 0 & 0.281\tabularnewline
priorfyocc & -1.1645963421 & 0 & 0.383\tabularnewline
OLTV & -0.7754851517 & 0 & 0.217\tabularnewline
Division & -0.8479998614 & 0 & 0.238\tabularnewline
interestonly & -0.9347473414 & 8.31589183e-158 & 0.057\tabularnewline
\bottomrule
\end{longtable}

The model coefficients, p-values and Gini coefficients of each of the
chosen explanatory variables indicate that these parameters explain the
reponse variable quite well.

\begin{center}\rule{0.5\linewidth}{\linethickness}\end{center}

5.1.3 Model coefficient comparison

The PD model is independently replicated by the validation team to
ensure correctness of the implementation. In order to assess whether the
replicated PD model yields the same output as the original model, the
risk drivers (independent parameters) and their model coefficients are
compared.

\begin{longtable}[]{@{}llll@{}}
\toprule
Parameter & Coef development & Coef validation & Coef
diff\tabularnewline
\midrule
\endhead
const & -4.4168461324 & -4.4168461324 & 0\tabularnewline
priorfydscr & -0.7618294827 & -0.7618294827 &
-2.220446049e-16\tabularnewline
mrfytdocc & -0.3871917877 & -0.3871917877 &
-9.992007222e-16\tabularnewline
debt\_yield\_p1 & -0.1510340457 & -0.1510340457 & 0\tabularnewline
priorfyocc & -0.4915827429 & -0.4915827429 &
-5.551115123e-17\tabularnewline
OLTV & -0.310497772 & -0.310497772 & -8.881784197e-16\tabularnewline
Division & -0.5789169907 & -0.5789169907 & 0\tabularnewline
interestonly & -0.3459055608 & -0.3459055608 & 0\tabularnewline
\bottomrule
\end{longtable}

The differences between the results of the independent model replication
and the results of the development team, both for the model coefficients
and their standard deviation, indicate the correctness of the
implementation of the PD model. In case these differences are not equal
to zero, further investigation is needed into why this would be the case
(e.g., implementation issues, specifications of the underlying data set,
parameterization of the PD model, etc).

\begin{center}\rule{0.5\linewidth}{\linethickness}\end{center}

5.1.3.2 Scoring

Based on the forecasted probability of default, each facility is assiged
to a rating. Below the distribution of the ratings for the train and
test dataset.

Train dataset distribution

Test dataset distribution

\begin{center}\rule{0.5\linewidth}{\linethickness}\end{center}

5.2 PD model stress test

5.2.1 Kolmogorov--Smirnov Test

The KS statistic is used to measure the discriminatory power of the PD
model. It is defined as the maximum difference between the cumulative
percentage of good samples (i.e., non-defaulters) and the cumulative
percentage of bad samples (i.e., defaulters). A higher KS value implies
a good fit of the model.

The test returns a KS value.

\begin{longtable}[]{@{}ll@{}}
\toprule
Ks statistic train & Ks statistic test\tabularnewline
\midrule
\endhead
0.4219 & 0.3811\tabularnewline
\bottomrule
\end{longtable}

For training sample, the KS value is 0.4219, which is higher than the
threshold of 0.4, indicating that discriminatory power of the model is
good. For testing sample, the KS value is 0.3811, which is lower than
the threshold of 0.4, indicating that discriminatory power of the model
is not sufficient.

\begin{center}\rule{0.5\linewidth}{\linethickness}\end{center}

5.2.2 Variance Inflation Factor

This test is to assess the multicollinearity among the risk drivers. The
variance inflation factor (VIF) quantifies the severity of
multicollinearity in an regression analysis caused by correlation
between multiple `independent variables' in a model (i.e., the risk
drivers). It is a measure of how much the variance of an estimated
regression coefficient is inflated because of multicollinearity in the
model. High levels of multicollinearity results in unstable parameter
estimates and renders the model statistically invalid.

The test returns VIF value for each variable listed in the table below.

\begin{longtable}[]{@{}ll@{}}
\toprule
Variable & Vif\tabularnewline
\midrule
\endhead
priorfydscr\_transformed & 1.3352\tabularnewline
mrfytdocc\_transformed & 1.6708\tabularnewline
debt\_yield\_p1\_transformed & 1.364\tabularnewline
priorfyocc\_transformed & 1.6813\tabularnewline
OLTV\_transformed & 1.2259\tabularnewline
Division\_transformed & 1.0953\tabularnewline
interestonly\_transformed & 1.045\tabularnewline
\bottomrule
\end{longtable}

As shown in the above table, all variables have a VIF lower than 4,
indicating the model does not suffer from multicollinearity.

\begin{center}\rule{0.5\linewidth}{\linethickness}\end{center}

5.2.3 Receiver Operating Characteristic curve

Receiver Operating Characteristic (ROC) is a curve in a unit square
which is used to assess the accuracy of the diagnostic (e.g.~rating)
system. The area under curve (AUC) is the area under the ROC curve.
Higher AUC means higher discriminatory power of model.

\begin{longtable}[]{@{}ll@{}}
\toprule
Area under curve & Gini coefficient\tabularnewline
\midrule
\endhead
0.7743 & 0.5487\tabularnewline
\bottomrule
\end{longtable}

As shown in the above table, the AUC value is higher than 0.7 and the
Gini value of the model is higher than 0.4, indicating that
discriminatory power of the model is good.

\begin{center}\rule{0.5\linewidth}{\linethickness}\end{center}

5.2.4 Normal Test

The aim of the test is to evaluate the adequacy of observed default
rates comparing to predicted PD. The test is performed on the level of
rating grade. It is applied under the assumption that the mean default
rate does not vary too much over time and that default events in
different years are independent. The normal test is motivated by the
Central Limit Theorem and is based on a normal approximation of the
distribution of the time-averaged default rates. With high enough
(\textgreater{}30) number (n) of observations which are independent,
Point in Time (PIT) observed default rate (ODR) is expected to be
normally distributed. The algorithm of the test is to calculate
time-weighted (or simple arithmetic) average of all PiT ODRs, calculate
standard error of this statistic and derive confidence interval for it.

The test returns a table of p values and a brief interpretation of the
test result for each rating grade.

\begin{longtable}[]{@{}lllll@{}}
\toprule
\begin{minipage}[b]{0.12\columnwidth}\raggedright
Rating class\strut
\end{minipage} & \begin{minipage}[b]{0.26\columnwidth}\raggedright
Predicted pd upper boundary\strut
\end{minipage} & \begin{minipage}[b]{0.12\columnwidth}\raggedright
Default rate\strut
\end{minipage} & \begin{minipage}[b]{0.19\columnwidth}\raggedright
Normal test p value\strut
\end{minipage} & \begin{minipage}[b]{0.18\columnwidth}\raggedright
Normal test result\strut
\end{minipage}\tabularnewline
\midrule
\endhead
\begin{minipage}[t]{0.12\columnwidth}\raggedright
1\strut
\end{minipage} & \begin{minipage}[t]{0.26\columnwidth}\raggedright
0.1\%\strut
\end{minipage} & \begin{minipage}[t]{0.12\columnwidth}\raggedright
0.13\%\strut
\end{minipage} & \begin{minipage}[t]{0.19\columnwidth}\raggedright
0.1063\strut
\end{minipage} & \begin{minipage}[t]{0.18\columnwidth}\raggedright
Acceptable\strut
\end{minipage}\tabularnewline
\begin{minipage}[t]{0.12\columnwidth}\raggedright
2\strut
\end{minipage} & \begin{minipage}[t]{0.26\columnwidth}\raggedright
0.26\%\strut
\end{minipage} & \begin{minipage}[t]{0.12\columnwidth}\raggedright
0.22\%\strut
\end{minipage} & \begin{minipage}[t]{0.19\columnwidth}\raggedright
0.1439\strut
\end{minipage} & \begin{minipage}[t]{0.18\columnwidth}\raggedright
Acceptable\strut
\end{minipage}\tabularnewline
\begin{minipage}[t]{0.12\columnwidth}\raggedright
3\strut
\end{minipage} & \begin{minipage}[t]{0.26\columnwidth}\raggedright
0.42\%\strut
\end{minipage} & \begin{minipage}[t]{0.12\columnwidth}\raggedright
0.42\%\strut
\end{minipage} & \begin{minipage}[t]{0.19\columnwidth}\raggedright
0.1088\strut
\end{minipage} & \begin{minipage}[t]{0.18\columnwidth}\raggedright
Acceptable\strut
\end{minipage}\tabularnewline
\begin{minipage}[t]{0.12\columnwidth}\raggedright
4\strut
\end{minipage} & \begin{minipage}[t]{0.26\columnwidth}\raggedright
0.56\%\strut
\end{minipage} & \begin{minipage}[t]{0.12\columnwidth}\raggedright
0.53\%\strut
\end{minipage} & \begin{minipage}[t]{0.19\columnwidth}\raggedright
0.1982\strut
\end{minipage} & \begin{minipage}[t]{0.18\columnwidth}\raggedright
Acceptable\strut
\end{minipage}\tabularnewline
\begin{minipage}[t]{0.12\columnwidth}\raggedright
5\strut
\end{minipage} & \begin{minipage}[t]{0.26\columnwidth}\raggedright
0.71\%\strut
\end{minipage} & \begin{minipage}[t]{0.12\columnwidth}\raggedright
0.7\%\strut
\end{minipage} & \begin{minipage}[t]{0.19\columnwidth}\raggedright
0.2359\strut
\end{minipage} & \begin{minipage}[t]{0.18\columnwidth}\raggedright
Acceptable\strut
\end{minipage}\tabularnewline
\begin{minipage}[t]{0.12\columnwidth}\raggedright
6\strut
\end{minipage} & \begin{minipage}[t]{0.26\columnwidth}\raggedright
0.93\%\strut
\end{minipage} & \begin{minipage}[t]{0.12\columnwidth}\raggedright
0.91\%\strut
\end{minipage} & \begin{minipage}[t]{0.19\columnwidth}\raggedright
0.2364\strut
\end{minipage} & \begin{minipage}[t]{0.18\columnwidth}\raggedright
Acceptable\strut
\end{minipage}\tabularnewline
\begin{minipage}[t]{0.12\columnwidth}\raggedright
7\strut
\end{minipage} & \begin{minipage}[t]{0.26\columnwidth}\raggedright
1.12\%\strut
\end{minipage} & \begin{minipage}[t]{0.12\columnwidth}\raggedright
0.7\%\strut
\end{minipage} & \begin{minipage}[t]{0.19\columnwidth}\raggedright
0.2665\strut
\end{minipage} & \begin{minipage}[t]{0.18\columnwidth}\raggedright
Acceptable\strut
\end{minipage}\tabularnewline
\begin{minipage}[t]{0.12\columnwidth}\raggedright
8\strut
\end{minipage} & \begin{minipage}[t]{0.26\columnwidth}\raggedright
1.41\%\strut
\end{minipage} & \begin{minipage}[t]{0.12\columnwidth}\raggedright
1.1\%\strut
\end{minipage} & \begin{minipage}[t]{0.19\columnwidth}\raggedright
0.4014\strut
\end{minipage} & \begin{minipage}[t]{0.18\columnwidth}\raggedright
Acceptable\strut
\end{minipage}\tabularnewline
\begin{minipage}[t]{0.12\columnwidth}\raggedright
9\strut
\end{minipage} & \begin{minipage}[t]{0.26\columnwidth}\raggedright
2.56\%\strut
\end{minipage} & \begin{minipage}[t]{0.12\columnwidth}\raggedright
1.91\%\strut
\end{minipage} & \begin{minipage}[t]{0.19\columnwidth}\raggedright
0.4329\strut
\end{minipage} & \begin{minipage}[t]{0.18\columnwidth}\raggedright
Acceptable\strut
\end{minipage}\tabularnewline
\begin{minipage}[t]{0.12\columnwidth}\raggedright
10\strut
\end{minipage} & \begin{minipage}[t]{0.26\columnwidth}\raggedright
15\%\strut
\end{minipage} & \begin{minipage}[t]{0.12\columnwidth}\raggedright
5.37\%\strut
\end{minipage} & \begin{minipage}[t]{0.19\columnwidth}\raggedright
0.0627\strut
\end{minipage} & \begin{minipage}[t]{0.18\columnwidth}\raggedright
Acceptable\strut
\end{minipage}\tabularnewline
\bottomrule
\end{longtable}

As shown in the above table, the p-value of Normal test for all segments
is higher than 0.05, which means the null hypothesis cannot be rejected
and there is evidence that the predicted PD does not deviate from the
long run average of observed values. Therefore, the accuracy of the
model is adequate.

\begin{center}\rule{0.5\linewidth}{\linethickness}\end{center}

\begin{center}\rule{0.5\linewidth}{\linethickness}\end{center}

5.2 PD model stress test

5.2.1 Kolmogorov--Smirnov Test

The KS statistic is used to measure the discriminatory power of the PD
model. It is defined as the maximum difference between the cumulative
percentage of good samples (i.e., non-defaulters) and the cumulative
percentage of bad samples (i.e., defaulters). A higher KS value implies
a good fit of the model.

The test returns a KS value.

\begin{longtable}[]{@{}ll@{}}
\toprule
Ks statistic train & Ks statistic test\tabularnewline
\midrule
\endhead
0.4219 & 0.3811\tabularnewline
\bottomrule
\end{longtable}

For training sample, the KS value is 0.4219, which is higher than the
threshold of 0.4, indicating that discriminatory power of the model is
good. For testing sample, the KS value is 0.3811, which is lower than
the threshold of 0.4, indicating that discriminatory power of the model
is not sufficient.

\begin{center}\rule{0.5\linewidth}{\linethickness}\end{center}

5.2.2 Variance Inflation Factor

This test is to assess the multicollinearity among the risk drivers. The
variance inflation factor (VIF) quantifies the severity of
multicollinearity in an regression analysis caused by correlation
between multiple `independent variables' in a model (i.e., the risk
drivers). It is a measure of how much the variance of an estimated
regression coefficient is inflated because of multicollinearity in the
model. High levels of multicollinearity results in unstable parameter
estimates and renders the model statistically invalid.

The test returns VIF value for each variable listed in the table below.

\begin{longtable}[]{@{}ll@{}}
\toprule
Variable & Vif\tabularnewline
\midrule
\endhead
priorfydscr\_transformed & 1.3352\tabularnewline
mrfytdocc\_transformed & 1.6708\tabularnewline
debt\_yield\_p1\_transformed & 1.364\tabularnewline
priorfyocc\_transformed & 1.6813\tabularnewline
OLTV\_transformed & 1.2259\tabularnewline
Division\_transformed & 1.0953\tabularnewline
interestonly\_transformed & 1.045\tabularnewline
\bottomrule
\end{longtable}

As shown in the above table, all variables have a VIF lower than 4,
indicating the model does not suffer from multicollinearity.

\begin{center}\rule{0.5\linewidth}{\linethickness}\end{center}

5.2.3 Receiver Operating Characteristic curve

Receiver Operating Characteristic (ROC) is a curve in a unit square
which is used to assess the accuracy of the diagnostic (e.g.~rating)
system. The area under curve (AUC) is the area under the ROC curve.
Higher AUC means higher discriminatory power of model.

\begin{longtable}[]{@{}ll@{}}
\toprule
Area under curve & Gini coefficient\tabularnewline
\midrule
\endhead
0.7743 & 0.5487\tabularnewline
\bottomrule
\end{longtable}

As shown in the above table, the AUC value is higher than 0.7 and the
Gini value of the model is higher than 0.4, indicating that
discriminatory power of the model is good.

\begin{center}\rule{0.5\linewidth}{\linethickness}\end{center}

5.2.4 Normal Test

The aim of the test is to evaluate the adequacy of observed default
rates comparing to predicted PD. The test is performed on the level of
rating grade. It is applied under the assumption that the mean default
rate does not vary too much over time and that default events in
different years are independent. The normal test is motivated by the
Central Limit Theorem and is based on a normal approximation of the
distribution of the time-averaged default rates. With high enough
(\textgreater{}30) number (n) of observations which are independent,
Point in Time (PIT) observed default rate (ODR) is expected to be
normally distributed. The algorithm of the test is to calculate
time-weighted (or simple arithmetic) average of all PiT ODRs, calculate
standard error of this statistic and derive confidence interval for it.

The test returns a table of p values and a brief interpretation of the
test result for each rating grade.

\begin{longtable}[]{@{}lllll@{}}
\toprule
\begin{minipage}[b]{0.12\columnwidth}\raggedright
Rating class\strut
\end{minipage} & \begin{minipage}[b]{0.26\columnwidth}\raggedright
Predicted pd upper boundary\strut
\end{minipage} & \begin{minipage}[b]{0.12\columnwidth}\raggedright
Default rate\strut
\end{minipage} & \begin{minipage}[b]{0.19\columnwidth}\raggedright
Normal test p value\strut
\end{minipage} & \begin{minipage}[b]{0.18\columnwidth}\raggedright
Normal test result\strut
\end{minipage}\tabularnewline
\midrule
\endhead
\begin{minipage}[t]{0.12\columnwidth}\raggedright
1\strut
\end{minipage} & \begin{minipage}[t]{0.26\columnwidth}\raggedright
0.1\%\strut
\end{minipage} & \begin{minipage}[t]{0.12\columnwidth}\raggedright
0.13\%\strut
\end{minipage} & \begin{minipage}[t]{0.19\columnwidth}\raggedright
0.1063\strut
\end{minipage} & \begin{minipage}[t]{0.18\columnwidth}\raggedright
Acceptable\strut
\end{minipage}\tabularnewline
\begin{minipage}[t]{0.12\columnwidth}\raggedright
2\strut
\end{minipage} & \begin{minipage}[t]{0.26\columnwidth}\raggedright
0.26\%\strut
\end{minipage} & \begin{minipage}[t]{0.12\columnwidth}\raggedright
0.22\%\strut
\end{minipage} & \begin{minipage}[t]{0.19\columnwidth}\raggedright
0.1439\strut
\end{minipage} & \begin{minipage}[t]{0.18\columnwidth}\raggedright
Acceptable\strut
\end{minipage}\tabularnewline
\begin{minipage}[t]{0.12\columnwidth}\raggedright
3\strut
\end{minipage} & \begin{minipage}[t]{0.26\columnwidth}\raggedright
0.42\%\strut
\end{minipage} & \begin{minipage}[t]{0.12\columnwidth}\raggedright
0.42\%\strut
\end{minipage} & \begin{minipage}[t]{0.19\columnwidth}\raggedright
0.1088\strut
\end{minipage} & \begin{minipage}[t]{0.18\columnwidth}\raggedright
Acceptable\strut
\end{minipage}\tabularnewline
\begin{minipage}[t]{0.12\columnwidth}\raggedright
4\strut
\end{minipage} & \begin{minipage}[t]{0.26\columnwidth}\raggedright
0.56\%\strut
\end{minipage} & \begin{minipage}[t]{0.12\columnwidth}\raggedright
0.53\%\strut
\end{minipage} & \begin{minipage}[t]{0.19\columnwidth}\raggedright
0.1982\strut
\end{minipage} & \begin{minipage}[t]{0.18\columnwidth}\raggedright
Acceptable\strut
\end{minipage}\tabularnewline
\begin{minipage}[t]{0.12\columnwidth}\raggedright
5\strut
\end{minipage} & \begin{minipage}[t]{0.26\columnwidth}\raggedright
0.71\%\strut
\end{minipage} & \begin{minipage}[t]{0.12\columnwidth}\raggedright
0.7\%\strut
\end{minipage} & \begin{minipage}[t]{0.19\columnwidth}\raggedright
0.2359\strut
\end{minipage} & \begin{minipage}[t]{0.18\columnwidth}\raggedright
Acceptable\strut
\end{minipage}\tabularnewline
\begin{minipage}[t]{0.12\columnwidth}\raggedright
6\strut
\end{minipage} & \begin{minipage}[t]{0.26\columnwidth}\raggedright
0.93\%\strut
\end{minipage} & \begin{minipage}[t]{0.12\columnwidth}\raggedright
0.91\%\strut
\end{minipage} & \begin{minipage}[t]{0.19\columnwidth}\raggedright
0.2364\strut
\end{minipage} & \begin{minipage}[t]{0.18\columnwidth}\raggedright
Acceptable\strut
\end{minipage}\tabularnewline
\begin{minipage}[t]{0.12\columnwidth}\raggedright
7\strut
\end{minipage} & \begin{minipage}[t]{0.26\columnwidth}\raggedright
1.12\%\strut
\end{minipage} & \begin{minipage}[t]{0.12\columnwidth}\raggedright
0.7\%\strut
\end{minipage} & \begin{minipage}[t]{0.19\columnwidth}\raggedright
0.2665\strut
\end{minipage} & \begin{minipage}[t]{0.18\columnwidth}\raggedright
Acceptable\strut
\end{minipage}\tabularnewline
\begin{minipage}[t]{0.12\columnwidth}\raggedright
8\strut
\end{minipage} & \begin{minipage}[t]{0.26\columnwidth}\raggedright
1.41\%\strut
\end{minipage} & \begin{minipage}[t]{0.12\columnwidth}\raggedright
1.1\%\strut
\end{minipage} & \begin{minipage}[t]{0.19\columnwidth}\raggedright
0.4014\strut
\end{minipage} & \begin{minipage}[t]{0.18\columnwidth}\raggedright
Acceptable\strut
\end{minipage}\tabularnewline
\begin{minipage}[t]{0.12\columnwidth}\raggedright
9\strut
\end{minipage} & \begin{minipage}[t]{0.26\columnwidth}\raggedright
2.56\%\strut
\end{minipage} & \begin{minipage}[t]{0.12\columnwidth}\raggedright
1.91\%\strut
\end{minipage} & \begin{minipage}[t]{0.19\columnwidth}\raggedright
0.4329\strut
\end{minipage} & \begin{minipage}[t]{0.18\columnwidth}\raggedright
Acceptable\strut
\end{minipage}\tabularnewline
\begin{minipage}[t]{0.12\columnwidth}\raggedright
10\strut
\end{minipage} & \begin{minipage}[t]{0.26\columnwidth}\raggedright
15\%\strut
\end{minipage} & \begin{minipage}[t]{0.12\columnwidth}\raggedright
5.37\%\strut
\end{minipage} & \begin{minipage}[t]{0.19\columnwidth}\raggedright
0.0627\strut
\end{minipage} & \begin{minipage}[t]{0.18\columnwidth}\raggedright
Acceptable\strut
\end{minipage}\tabularnewline
\bottomrule
\end{longtable}

As shown in the above table, the p-value of Normal test for all segments
is higher than 0.05, which means the null hypothesis cannot be rejected
and there is evidence that the predicted PD does not deviate from the
long run average of observed values. Therefore, the accuracy of the
model is adequate.
